%%%%%%%%%%%%%%%%%%%%%%%%%%%%%%%%%%%%%%%%%
% Article EcoFoG
% Version 2.1 (23/10/2017)
%
% adapté de :
% Stylish Article
% LaTeX Template
% Version 1.0 (31/1/13)
%
% This template has been downloaded from:
% http://www.LaTeXTemplates.com
%
% Original author:
% Mathias Legrand (legrand.mathias@gmail.com)
%
% License:
% CC BY-NC-SA 3.0 (http://creativecommons.org/licenses/by-nc-sa/3.0/)
%
%%%%%%%%%%%%%%%%%%%%%%%%%%%%%%%%%%%%%%%%%


%----------------------------------------------------------------------------------------
%	PACKAGES AND OTHER DOCUMENT CONFIGURATIONS
%----------------------------------------------------------------------------------------

\documentclass[fleqn,10pt]{ArtEcoFoG} % Document font size and equations flushed left

\setcounter{tocdepth}{3} % Show only three levels in the table of contents section: sections, subsections and subsubsections


% Pandoc environments
\usepackage{framed}
\usepackage{fancyvrb}
\providecommand{\tightlist}{%
  \setlength{\itemsep}{0pt}\setlength{\parskip}{0pt}}
\newcommand{\VerbBar}{|}
\newcommand{\VERB}{\Verb[commandchars=\\\{\}]}
\DefineVerbatimEnvironment{Highlighting}{Verbatim}{commandchars=\\\{\}, fontsize=\scriptsize} % Code R
\definecolor{shadecolor}{RGB}{248,248,248}
\newenvironment{Shaded}{\begin{snugshade}}{\end{snugshade}}
\newcommand{\KeywordTok}[1]{\textcolor[rgb]{0.13,0.29,0.53}{\textbf{{#1}}}}
\newcommand{\DataTypeTok}[1]{\textcolor[rgb]{0.13,0.29,0.53}{{#1}}}
\newcommand{\DecValTok}[1]{\textcolor[rgb]{0.00,0.00,0.81}{{#1}}}
\newcommand{\BaseNTok}[1]{\textcolor[rgb]{0.00,0.00,0.81}{{#1}}}
\newcommand{\FloatTok}[1]{\textcolor[rgb]{0.00,0.00,0.81}{{#1}}}
\newcommand{\ConstantTok}[1]{\textcolor[rgb]{0.00,0.00,0.00}{{#1}}}
\newcommand{\CharTok}[1]{\textcolor[rgb]{0.31,0.60,0.02}{{#1}}}
\newcommand{\SpecialCharTok}[1]{\textcolor[rgb]{0.00,0.00,0.00}{{#1}}}
\newcommand{\StringTok}[1]{\textcolor[rgb]{0.31,0.60,0.02}{{#1}}}
\newcommand{\VerbatimStringTok}[1]{\textcolor[rgb]{0.31,0.60,0.02}{{#1}}}
\newcommand{\SpecialStringTok}[1]{\textcolor[rgb]{0.31,0.60,0.02}{{#1}}}
\newcommand{\ImportTok}[1]{{#1}}
\newcommand{\CommentTok}[1]{\textcolor[rgb]{0.56,0.35,0.01}{\textit{{#1}}}}
\newcommand{\DocumentationTok}[1]{\textcolor[rgb]{0.56,0.35,0.01}{\textbf{\textit{{#1}}}}}
\newcommand{\AnnotationTok}[1]{\textcolor[rgb]{0.56,0.35,0.01}{\textbf{\textit{{#1}}}}}
\newcommand{\CommentVarTok}[1]{\textcolor[rgb]{0.56,0.35,0.01}{\textbf{\textit{{#1}}}}}
\newcommand{\OtherTok}[1]{\textcolor[rgb]{0.56,0.35,0.01}{{#1}}}
\newcommand{\FunctionTok}[1]{\textcolor[rgb]{0.00,0.00,0.00}{{#1}}}
\newcommand{\VariableTok}[1]{\textcolor[rgb]{0.00,0.00,0.00}{{#1}}}
\newcommand{\ControlFlowTok}[1]{\textcolor[rgb]{0.13,0.29,0.53}{\textbf{{#1}}}}
\newcommand{\OperatorTok}[1]{\textcolor[rgb]{0.81,0.36,0.00}{\textbf{{#1}}}}
\newcommand{\BuiltInTok}[1]{{#1}}
\newcommand{\ExtensionTok}[1]{{#1}}
\newcommand{\PreprocessorTok}[1]{\textcolor[rgb]{0.56,0.35,0.01}{\textit{{#1}}}}
\newcommand{\AttributeTok}[1]{\textcolor[rgb]{0.77,0.63,0.00}{{#1}}}
\newcommand{\RegionMarkerTok}[1]{{#1}}
\newcommand{\InformationTok}[1]{\textcolor[rgb]{0.56,0.35,0.01}{\textbf{\textit{{#1}}}}}
\newcommand{\WarningTok}[1]{\textcolor[rgb]{0.56,0.35,0.01}{\textbf{\textit{{#1}}}}}
\newcommand{\AlertTok}[1]{\textcolor[rgb]{0.94,0.16,0.16}{{#1}}}
\newcommand{\ErrorTok}[1]{\textcolor[rgb]{0.64,0.00,0.00}{\textbf{{#1}}}}
\newcommand{\NormalTok}[1]{{#1}}
\usepackage{longtable,booktabs}
\usepackage{caption}
% These lines are needed to make table captions work with longtable:
\makeatletter
\def\fnum@table{\tablename~\thetable}
\makeatother
% longtable 2 columns
% https://tex.stackexchange.com/questions/161431/how-to-solve-longtable-is-not-in-1-column-mode-error
\makeatletter
\let\oldlt\longtable
\let\endoldlt\endlongtable
\def\longtable{\@ifnextchar[\longtable@i \longtable@ii}
\def\longtable@i[#1]{\begin{figure}[t]
\onecolumn
\begin{minipage}{0.5\textwidth}\scriptsize
\oldlt[#1]
}
\def\longtable@ii{\begin{figure}[t]
\onecolumn
\begin{minipage}{0.5\textwidth}\scriptsize
\oldlt
}
\def\endlongtable{\endoldlt
\end{minipage}
\twocolumn
\end{figure}}
\makeatother

\usepackage{graphicx,grffile}
\makeatletter
\def\maxwidth{\ifdim\Gin@nat@width>\linewidth\linewidth\else\Gin@nat@width\fi}
\def\maxheight{\ifdim\Gin@nat@height>\textheight0.8\textheight\else\Gin@nat@height\fi}
\makeatother
% Scale images if necessary, so that they will not overflow the page
% margins by default, and it is still possible to overwrite the defaults
% using explicit options in \includegraphics[width, height, ...]{}
\setkeys{Gin}{width=\maxwidth,height=\maxheight,keepaspectratio}

% User-adder preamble
\usepackage{textcomp} \DeclareUnicodeCharacter{B0}{\textdegree}
\hyphenation{sa-plings} \usepackage{longtable,tabu}

%----------------------------------------------------------------------------------------
%	ARTICLE INFORMATION
%----------------------------------------------------------------------------------------

\JournalInfo{\ }
\Archive{\ }

\PaperTitle{Inescapable Taxonomists: Workable Biodiversity Management Based on a
Minimum Field Work} % Article title

\Authors{
Ariane Mirabel\textsuperscript{1*}\\ Eric Marcon\textsuperscript{1}\\ Bruno Hérault\textsuperscript{2}
} % Authors
\affiliation{
\textsuperscript{1}UMR EcoFoG, AgroParistech, CNRS, Cirad, INRA, Université des Antilles,
Université de Guyane.\\ \hspace{1em} Campus Agronomique, 97310 Kourou, France.\\\textsuperscript{2}INPHB (Institut National Ploytechnique Félix Houphoüet Boigny)\\ \hspace{1em} Yamoussoukro, Ivory Coast
}
\affiliation{*\textbf{Corresponding author}: ariane.mirabel@ecofog.gf, https://github.com/ArianeMirabel} % Corresponding author

\Keywords{Biodiversity Measurement, Tree Community, Neotropical Forests, Botanical Uncertainty Propagation, Bayesian Estimator} % Keywords - if you don't want any simply remove all the text between the curly brackets
\newcommand{\keywordname}{Keywords} % Defines the keywords heading name

%----------------------------------------------------------------------------------------
%	ABSTRACT
%----------------------------------------------------------------------------------------

\Abstract{
Assess the fate of Neotropical forests requires to accurately measures
are the base of reliable foret monitoring. The costs of botanical
inventories and the taxonomic complexity of Neotropical forests make
forest inventories in vernacular names the most efficient approach
today, although these hold high botanical uncertainty and limit the
accuracy of diversity measures. Several methods were proposed to
compensate these botanical uncertainties but none reliably assessed
functional and fine-scale diversity surveys. We developed a polyvalent
diversity estimator workable in numerous specific cases based on the
propagation of botanical uncertainties. The estimator was calibrated
with a large neotropical inventory and the simulations of uncertainty
and sampling effort gradients allowed to determined an ideal inventory
protocol optimizing the costs and the accuracy of forest inventories.
Our study first advocated of necessity of real inventories and the
inescapable recourse to taxonomists to ensure reliable diversity
estimations. An ideal inventory protocol based on a sampling effort of
XX trees and on an identification effort of 80\% of the species was
identified and ensured diversity estimations with a 10\% error.
}

%----------------------------------------------------------------------------------------

\begin{document}

\selectlanguage{english}

\flushbottom % Makes all text pages the same height

\maketitle % Print the title and abstract box

\tableofcontents % Print the contents section

\thispagestyle{empty} % Removes page numbering from the first page

%----------------------------------------------------------------------------------------
%	ARTICLE CONTENTS
%----------------------------------------------------------------------------------------


\section{Introduction}\label{introduction}

The variety of tree species, their assemblages in space and their
dynamics in time are determinant of forests productivity and functioning
\citep{Cardinale2012}. Preserve tree diversity is crucial to maintain
forests functioning and services, specifically in hyper-diverse tropical
forests where the biodiversity is as threatened as it is valuable and
unexplored \citep{Barlow2018}. Handling the conservation and management
of tree diversity requires setting sensible protection areas and
sustainable forest management calibrated according to diversity patterns
in space and time and their determinants
\citep{Margules2000, Purvis2000, Gibson2011a, FAO2014, Sist2015}.

Correctly measure, map and manage forests biodiversity require accurate
and large forest monitoring. The precision of forest inventories,
though, is often limited by their significant cost in terms of time,
money, and logistic \citep{Feeley2011, Valencia2013}. Sampling methods
were optimized to minimize these costs and maximize inventory
accuracy.Some approaches would restrict inventories to some DBH or
height classes, to specific taxa, or would opt for inventories at family
or genus level. These methods efficiently translated biodiversity
patterns at regional scales and along wide ecological gradients
\citep{Steege2000, Higgins2004, Rejou-Mechain2011, Pos2014}. However,
these methods were either limited to small areas (under 1ha), sometimes
remained biased or holding significant uncertainty, and usually proved
limited to detect subtle diversity aspects and to desentangle richness
from equitability parameters
\citetext{\citealp{Phillips2003a}; \citealp{Valencia2013}; \citealp[
]{Guitet2014b}; \citealp{Vellend2008}; \citealp{Prance1994}}. Another
approach proposed to use inventories in vernacular names instead of
botanical species. Vernacular names indeed are easier to attribute, more
common and usually do not require vouchers collection or posterior
botanical identification. The reliability of vernacular names may be
high at genus level, but this proved highly variable across tropical
regions: while this reliability was estimated around 60-70\% in French
Guiana \citep{Hawes2012, Guitet2014b} to ranges from 32\% to 67\% in
Central Africa \citep{Rejou-Mechain2011}. The multiple and variable
associations between botanical and vernacular names then entail
significant botanical uncertainties that should not be ignored
\citep{Oldeman1968}. Besides, rough vernacular inventories would not
allow functional and phylogenetic approaches, that require
identification at the botanical species to comply with phylogenetic and
functional database. However the approach through vernacular names
deserves further attention. First, it gives the opportunity to analyze
pre-logging inventories conducted in large areas by logging companies.
Second, as exhaustive inventories, they allow some post-process based on
vernacular/botanical names association and allow the building of
reliable diversity estimators
\citep{TerSteege2006, Feldpausch2006, Rejou-Mechain2008, Rejou-Mechain2011}.
Following this idea \citet{Guitet2014b} proposed a framework propagating
vernacular names taxonomic uncertainties in diversity measures. The
propagation framework was based on Monte-Carlo processes estimating
forest diversity from the vernacular-botanical name association. These
association combined prior information from both general taxa-abundance
correspondence table \citep{Molino2009} and reference field inventories.
The framework successfully rendered the ranking of plots diversity, but
remained restricted to large environmental gradient and for highly
different communities \citep{Guitet2014b, Guitet2013}. In this study we
offer to refine this framework and adapt it to diversity estimation at
smaller spatial scales. The following diversity estimator is based on
the specific case of the studied community and the inventory protocol.
The diversity estimator besides suits all inventories whatever the ratio
of botanical determination, \emph{i.e.} ratio of vernacular compared to
botanical names. It besides suits experimental specific as well as
pre-logging inventories where only the commercial or most recognizable
species are identified at species level.

Such diversity estimator allows maximizing the accuracy of diversity
measures while minimizing the sampling effort, \emph{i.e.} the size of
inventoried communities and the number of accurately identified species.
In this perspective we thought to calibrate an ideal inventory protocol
optimized in terms of sampling effort and determination degree. From a
real inventory, with complete vernacular and botanical identifications,
we simulated ranges of sampling efforts and identification degrees along
which we examined the bias and variability of the diversity estimator.

In this study we \emph{(i)} redesigned a diversity estimator based on a
Bayesian framework accounting for both general taxa-association tables
and specific field inventories, and \emph{(ii)} applied the estimator to
a real Neotropical forest inventory to determine the sampling effort and
determination degree of an ideal inventory protocol.

\section{Methods}\label{methods}

\subsection{Study community}\label{study-community}

We based our analyses on the inventory of a Neotropical rainforest, from
the Paracou Research Station in French Guiana (5°18'N and 52°53'W). The
experimental site stands in a lowland tropical rainforest with a flora
dominated by \emph{Fabaceae}, \emph{Chrysobalanaceae},
\emph{Lecythidaceae} and \emph{Sapotaceae} families. Mean mean annual
temperature is 26°C. and the mean annual precipitations average
\(2980 mm.y^-1\) (30-y period) with a 3-months dry season
(\(< 100 mm.months-1\)) from mid-August to mid-November and a one-month
dry season in March \citep{Wagner2011}. Elevation ranges between 5 and
50 m and soils correspond to thin acrisols over a layer of transformed
saprolite with low permeability, generating lateral drainage during
heavy rains \citep{IUSSWorkingGroupWRB2015}. We used the 2015 inventory
of six permanent plots of undisturbed forest (6.25ha each, 37.5ha
inventoried in total). During inventories trees are identified first
with a vernacular name assigned by the forest worker team, and afterward
with a scientific name assigned by botanists during regular botanical
campaigns. The community inventoried ancompasses 22 904 trees belonging
to 375 species and 63 families, identified by 290 different vernacular
names. The initial taxonomic uncertainty was 3\% of the community,
\emph{i.e.} the proportion of trees not identified with a botanical
name.

\subsection{Diversity measures}\label{diversity-measures}

Among the large panel of diversity indices we examined here the family
of q-generalized (Tsallis) entropy, widely adopted to assess all aspects
of taxonomic, functional and phylogenetic diversities. The Tsallis
diversity indices derive from a general formula, modulated by an order q
emphasizing species frequency \eqref{eq:TsallisEntropy}.

\begin{equation}
^qD = \sum_{i=1}^{N}{\left( p_i^q \right)^{\frac{1}{1-q}} }
\label{eq:TsallisEntropy}
\end{equation}

In the diversity formula, species relative abundance \(p_i\) in a
community of \(N\) species is raised at the power \(q\) that is the
order of the diversity. The higher the order \(q\), the higher the
emphasis on common vs.~rare species, so browsing a range of order \(q\)
corresponds assess a gradient balance between richness and evenness. The
formula retrieves species richness for \(q = 0\) , Shannon diversity for
\(q = 1\) where richness and evenness are equally accounted for and
Simpson diversity, that can be undestood as the diversity of common
species, for \(q = 2\). The Tsallis diversity indices would eventually
be converted into equivalent number of species in our framework. The
conversion in equivalent number of species, through Hill transformation,
allows understandable analysis and comparisons among communities
\citep{Hill1973, Keylock2005, Jost2006}.

\subsection{Diversity estimator}\label{diversity-estimator}

The estimation framework is based on the diversity distribution measured
on theoretical, fully determined communities. Theoretical inventories
are simulated 1 000 times from the real incomplete inventory, through
the replacement by a Monte-Carlo scheme of vernacular names by botanical
ones.

The vernacular-botanical replacement are based on the association
probability between each vernacular names and the botanical names
inventoried. For each vernacular name the association model follows a
multinomial distribution
\(M([s_1, s_2, …, s_N] ,[\alpha_1, \alpha_2,…, \alpha_N])\), with
\([\alpha_i]\) the association probability of botanical name \(s_i\)
with the vernacular name.

The association probability vectors \([\alpha_v]\) were determined with
a Bayesian framework based on the combination of botanical expertise and
observed associations. First, the estimation of \([\alpha_v]\) accounted
for prior information from experts' knowledge in the form of a general
taxa-association table listing all botanical names likely corresponding
to the vernacular name \(v\). From this general table, the probability
\(\lambda_i={}^1/m_v\) was attributed to each of the \(m_v\) botanical
names with a confirmed association with \(v\). When no association was
established the probability \(\lambda_i={}^\epsilon\big/_{N-m_v}\) was
attributed to the botanical name, with \(\epsilon\) standing for a
background noise set to 0.01 here. Second, the estimation of
\([\alpha_v]\) accounted for observed inventories giving real
association frequencies \(\phi_i\) between \(v\) and the \(m_v'\)
botanical names with observed association. Similarly, the association
probability \(\lambda_i={}^\epsilon\big/_{N-m_v'}\) was attributed to
botanical names with no observed association. The final \([\alpha_v]\)
distribution was modeled by a Multinomial-Dirichlet scheme combining the
two vectors \([\lambda^v]\) and \([\phi^v]\) \citep{McCarthy2007}.

To test the relevance of the general table and observed inventories
information, we tested a range of weighting \(w\). Assuming a
distribution of \([\phi^v]\) conditionally to \([\alpha^v]\) the
weighting returned the formula \eqref{eq:weighting}.

\begin{equation}
[\alpha_i^v]: 
\Big[\alpha_i^v | _{(1-w)\lambda_i^v ,w.\phi_i^v}\Big] =Dirichlet\Big((1-w)\phi_i^v+w.\lambda_i^v\Big)
\label{eq:weighting}
\end{equation}

When \(w=0\) only observed inventories were considered, when \(w=0.5\)
both information were equally accounted for and when \(w=1\) only the
general taxa-association table was considered.

\subsection{Simulation of determination and sampling effort
gradients}\label{simulation-of-determination-and-sampling-effort-gradients}

The simulation of a determination effort gradient, \emph{i.e.} an
increasing proportion of vernacular names among all identifications,
allowed to (i) examine the diversity estimator response to the
determination effort and (ii) determine the best input balance \(w\)
between general table and observed inventories for the estimator. The
indetermination gradient was simulated by ignoring an increasing number
of botanical identifications in the reference inventory. As rare species
had more chance to be undetermined (Kendall test,
\(\tau = -0.46, p < 10^-16\)), the trial of ignored determination
followed botanical names abundance (\(p_{undetermined}=f_i^{-0.1}\),
with \(f_i\) botanical name frequency).

The simulation of a sampling effort gradient, \emph{i.e.} an increasing
number of trees inventoried to compute the observed association
probability, assessed the response of the diversity estimator to
sampling effort. The simulated gradient ranged from 500 to 22 000 trees
randomly selected in the reference inventory.

Along both indetermination and sampling effort gradient we examine the
estimator bias, \emph{i.e.} the difference between the estimation and
the real diversity \citep{Baltanas2009}, the estimator variability,
\emph{i.e.} 95\% confidence interval.

\section{Results}\label{results}

\subsection{The reponse to determination effort, and the design of an
ideal
framework}\label{the-reponse-to-determination-effort-and-the-design-of-an-ideal-framework}

Along the indetermination gradient, when considering both general
taxa-association table and observed inventory the diversity was
increasingly overestimated (Fig. \ref{fig:Fig1}\emph{(a)}). This
overestimation increased with the order of diversity q, while it was not
significant for the richness (\(q=0\)), the overestimation reached 45\%
of the real diversity for Shannon diversity (\(q = 1\)) and it reached
57\% of the real diversity for the Simpson diversity (\(q = 2\)).

When only considering the general taxa-association table (Fig.
\ref{fig:Fig1}\emph{(b)}) the richness (\(q=0\)) was underestimated
(reaching a 50\% underestimation), while both Shannon and Simpson
diversities were overestimated (respectively reaching underestimations
of 67\% and 125\%).

When only considering the observed inventory (Fig.
\ref{fig:Fig1}\emph{(c)}) the estimator remained slightly biased but it
did not exceed 15\% of the real diversity for any order of diversity.

A bootstrap of the 100 simulations for each specific case and diversity
order showed a stabilization of variances after 60 simulations.

\begin{figure*}
\includegraphics[width=1\linewidth]{TaxonomicUncertainty_files/figure-latex/Fig1-1} \caption{Indices degradation along a taxonomic uncertainty gradient. 95\% envelopes of the Richness, Shannon and Simpson indices calculated through our propagation method along an uncertainty gradient from 0 to 100\% of undetermined species. In (a) Only expert prior is considered to compute the association frequencies, in (b) both expert and observation prior are equally accounted for in the propagation method and in (c) only the observation prior is considered.}\label{fig:Fig1}
\end{figure*}

\begin{quote}
r Fig2, out.width = `60\%', echo=FALSE,fig.cap=``Degradation along a
taxonomic uncertainty gradient of diversity estimated from a reference
field inventories of 2 000 trees. 95\% envelopes of the Richness,
Shannon and Simpson diversities calculated along an uncertainty gradient
from 0 to 100\% of undetermined species.''
\end{quote}

\subsection{Calibrating the sampling
effort}\label{calibrating-the-sampling-effort}

Along the sampling effort gradient from 500 to 22 000 trees, the
richness estimation remained underestimated but the estimator confidence
interval did not exceed 7\%. The Shannon and Simpson were less biased,
for 2 000 trees inventoried the Shannon diversity bias fell to 15\%
while the bias of Simpson estimator fell to 6\% (Fig. \ref{fig:Fig3}.

\begin{figure*}
\includegraphics[width=1\linewidth]{TaxonomicUncertainty_files/figure-latex/Fig3-1} \caption{Degradation along a sampling effort gradient of the Richness, Shannon and Simpson diversities estimated for the reference inventory in vernacular names. The propagation method to estimate the diversities is only based on the reference field inventory. Above plots correspond to the estimated diversity in equivalent number of species and below plots correspond to the relative bias of the estimation compared to the value of the reference field inventory. For both dashed lines represent the value of the reference field inventory and crosses and red lines respectively represent the mean, 0.05 and 0.95 quantiles estimated after 1000 iterations.}\label{fig:Fig3}
\end{figure*}

\section{Discussion}\label{discussion}

\subsection{Inescapable taxonomists}\label{inescapable-taxonomists}

The method developed in the line of \citet{Guitet2014b} to propagate the
taxonomic uncertainty of vernacular names in diversity mesures provided
a reliable estimator for diversity indices of different order. The use
the general taxonomic-association table proved to systematically
overestimate the diversity. Indeed in the general table
vernacular/botanical association probability were independent of
botanical names abundance, so rare vernacular were indifferently
replaced by rare or abundant botanical names. Randomly along the
simulations the abundance of rare species were inflated at the expense
abundant ones.

In contrast the use of observed inventories proved more reliable as it
accounted for botanical names abundance. The rescourse to taxonomists
and pre-inventories proved unavoidable to correctly estimate and
therefore manage forest biodiversity.

\subsection{Calibration of an optimized inventory
protocol}\label{calibration-of-an-optimized-inventory-protocol}

The response of the diversity estimator bias and variability allowed
determining the inventory protocol minimizing the determination ratio
and sampling effort while maximizing the estimator bias and variability.
The real richness proved difficult to assess whatever the sampling and
determination effort, as already suggested in previous analysis
comparing several inventory methods \citep{Higgins2004}. Still, although
the richness estimation remained biased the variation remained low and
allowed preserving communities ranking of with similar indetermination
ratio \citep{Vellend2008}.

Conversely, the Shannon and Simpson diversity estimations proved less
biased, and this key given their power to detect small time and spatial
scale diversity variations
\citep{Baraloto2012a, Berry2008a, Cannon1998, Plumptre1996}. From 2 000
pre-inventoried trees the Shannon and Simpson estimators respectively
displayed 12\% and 1\% uncertainty.

\section{Conclusion}\label{conclusion}

In this paper we developed a diversity estimator developed in this paper
propagating vernacular names' taxonomic uncertainty to the measure
tropical forest diversity. It proved reliable to estimate forest
diversity for all diversity order and highlighted the inescapable
rescourse to taxonomists and minimum real inventories. The response of
the estimator bias and variability along a sampling effort gradient
allowed optimizing an ideal inventory protocol. With an initial
reference botanical inventory of 2 000 trees Shannon and Simpson
diversities were estimated within 10\% and 1\% confidence intervals. The
diversity estimator allows integrating the specificity of the working
team and local forest structure and is thus adaptable to all specific
case.

%----------------------------------------------------------------------------------------
%	REFERENCE LIST
%----------------------------------------------------------------------------------------

\bibliographystyle{mee}
\makeatletter
% The filename has .bib extension the must be eliminated
\filename@parse{references.bib}
% parse stores the file name in base. Extension starts at the first dot, so don't use dots in file names.
\bibliography{\filename@base}
\makeatother


%----------------------------------------------------------------------------------------

\end{document}
